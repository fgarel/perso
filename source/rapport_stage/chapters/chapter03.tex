
Pour établir l'état initial, nous avons d'abord calculer les coordonnées des stations d'observations, puis les coordonnées des points fixes et des points cibles.

Pour établir les états suivants, nous nous basons sur les points fixes, c'est à dire qu'à partir des coordonnées des points fixes, nous déterminons les coordonnées des stations d'observations et des points cibles.

\section{Les Stations d'observation}

Les coordonnées d'une première station et son orientation ont été calculées dans le système Lambert 93 CC46.
Les coordonnées des autres stations ont été calculées en suivant les méthodes de topographie traditionnelles : cheminement, observation de points doubles, ...

En valeur absolue, les coordonnées de ces stations n'ont aucune incidence sur les distances relatives entre les points fixes et les points cibles.  

\section{Points fixes ou Points de référence}
Pour établir l'état initial, nous avons déterminer les coordonnées des points fixes à partir des coordonnées des stations.
Ces coordonnés ne seront plus modifiées lors des calculs pour les états suivants.

La précision des observations est de l'ordre de 2 à 3 millimètres, cependant nous avons choisi un nombre suffisant de points afin de moyenner ces erreurs d'observations.
Nous pouvons estimer que les coordonnées des points fixes, après calcul et compensation, ont une précision de 1 à 2 millimètres.
En effet, un point peut être observé plusieurs fois, à partir de deux ou trois stations différentes.
Aussi, le calcul des coordonnées est réalisé en utilisant un logiciel de calcul topométrique permettant la compensation en bloc.

\section{Points cibles}
En ce qui concerne l'état initial, les coordonnées des points cibles et des points fixes ont été calculés en même temps.

Par contre, concernant les états suivants, nous utilisons le logiciel de calcul topométrique en fixant les coordonnées des points fixes.

De même que pour les points fixes, nous pouvons estimer que :
\begin{itemize}
    \item la précision des observations est de l'ordre de 2 à 3 millimètres
    \item la précision des coordonnées est de l'ordre de 1 à 2 millimètres
\end{itemize}
Il en résulte que la précision des distances relatives entre les points fixes et les points cibles est de l'ordre de 2 à 3 millimètres
