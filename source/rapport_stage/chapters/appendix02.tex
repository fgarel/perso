
\section{Inclure des graphiques dans un rapport}
\subsection{Le problème}
Nous avons inserer dans notre document une photo.
Quand on compile le document tex, le message d'erreur mentionne l'absence d'un fichier photo.bb.
Ce type de fichier n'est donc pas créé de manière dynamique.
Il est possible de créer ce fichier avec l'utilitaire ebb, mais celui ci donne des bounding box
différentes que l'utilitaire identify

La solution consiste donc à jouer avec les options d'inclusion

\subsection{Les Options}

Nous allons travailler à partir de cette photo qui a été réalisée à l'aide de cette commande :
convert -resize 400x300 logo: test.jpg

\subsubsection{viewport}

L'option viewport fonctionne avec des coordonnées absolues.
Ainsi, {160 120 360 270} correspond à une zone partant du point {160 120}
et allant jusqu'au point {360 270}.
Cela correspond a une image de 200x150.

Sur la figure~\ref{viewport variant et bb constant}, page~\pageref{viewport variant et bb constant},
nous avons, sur les deux premières lignes, un decoupage sans déformation de l'image, 
tandis que, sur la troisième ligne, nous avons un découpage avec déformation de l'image.

% debut d'une premiere figure
\begin{figure}[!h]
  \centering
  %\includegraphics[width=\textwidth]{mydessin.pdf}
  \includegraphics[bb=0 0 400 300,height=3cm]{test.jpg}
  \caption{La photo originale}
  \label{Photo originale}
\end{figure}

% debut d'une seconde figure
\begin{figure}[h]
    \centering
    \begin{subfigure}[b]{0.3\textwidth}
        \includegraphics[viewport=0 0 400 300,bb=0 0 400 300,width=4cm,height=3cm,clip=true]{test.jpg}
        \caption{viewport=0 0 400 300\\bb=0 0 400 300}
        \label{essai_a}
    \end{subfigure}
    ~ %add desired spacing between images, e. g. ~, \quad, \qquad etc.
      %(or a blank line to force the subfigure onto a new line)
    \begin{subfigure}[b]{0.3\textwidth}
        \includegraphics[viewport=160 120 360 270,bb=0 0 400 300,width=4cm,height=3cm,clip=true]{test.jpg}
        \caption{viewport=160 120 360 270\\bb=0 0 400 300}%+200+150
        \label{essai_b}
    \end{subfigure}
    ~
    \begin{subfigure}[b]{0.3\textwidth}
        \includegraphics[viewport=160 120 320 240,bb=0 0 400 300,width=4cm,height=3cm,clip=true]{test.jpg}
        \caption{viewport=160 120 320 240\\bb=0 0 400 300}%+160+120
        \label{essai_c}
    \end{subfigure}
    \\
    \begin{subfigure}[b]{0.3\textwidth}
        \includegraphics[viewport=160 120 280 210,bb=0 0 400 300,width=4cm,height=3cm,clip=true]{test.jpg}
        \caption{viewport=160 120 280 210\\bb=0 0 400 300}%+120+90
        \label{essai_d}
    \end{subfigure}
    ~
    \begin{subfigure}[b]{0.3\textwidth}
        \includegraphics[viewport=160 120 240 180,bb=0 0 400 300,width=4cm,height=3cm,clip=true]{test.jpg}
        \caption{viewport=160 120 240 180\\bb=0 0 400 300}%+80+60
        \label{essai_e}
    \end{subfigure}
    ~
    \begin{subfigure}[b]{0.3\textwidth}
        \includegraphics[viewport=160 120 200 150,bb=0 0 400 300,width=4cm,height=3cm,clip=true]{test.jpg}
        \caption{viewport=160 120 200 150\\bb=0 0 400 300}%+40+30
        \label{essai_f}
    \end{subfigure}
    \\
    \begin{subfigure}[b]{0.3\textwidth}
        \includegraphics[viewport=160 120 240 210,bb=0 0 400 300,width=4cm,height=3cm,clip=true]{test.jpg}
        \caption{viewport=160 120 240 210\\bb=0 0 400 300}%+80+90
        \label{essai_g}
    \end{subfigure}
    ~
    \begin{subfigure}[b]{0.3\textwidth}
        \includegraphics[viewport=160 120 240 180,bb=0 0 400 300,width=4cm,height=3cm,clip=true]{test.jpg}
        \caption{viewport=160 120 240 180\\bb=0 0 400 300}%+80+60
        \label{essai_h}
    \end{subfigure}
    ~
    \begin{subfigure}[b]{0.3\textwidth}
        \includegraphics[viewport=160 120 240 150,bb=0 0 400 300,width=4cm,height=3cm,clip=true]{test.jpg}
        \caption{viewport=160 120 240 150\\bb=0 0 400 300}%+80+30
        \label{essai_i}
    \end{subfigure}
    \caption{Les options pour l'insertion d'une image :\\viewport variant et bb constant}%\label{fig-double}
    \label{viewport variant et bb constant}

\end{figure}


% debut d'une troisieme figure
\begin{figure}[h]
    \centering
    \begin{subfigure}[b]{0.3\textwidth}
        \includegraphics[viewport=0 0 400 300,bb=0 0 400 300,width=4cm,height=3cm,clip=true]{test.jpg}
        \caption{viewport=0 0 400 300\\bb=0 0 400 300}
        \label{essai_a}
    \end{subfigure}
    ~
    \begin{subfigure}[b]{0.3\textwidth}
        \includegraphics[viewport=0 0 400 300,bb=160 120 400 300,width=4cm,height=3cm,clip=true]{test.jpg}
        \caption{viewport=0 0 400 300\\bb=160 120 400 300}
        \label{essai_2}
    \end{subfigure}
    ~
    \begin{subfigure}[b]{0.3\textwidth}
        \includegraphics[viewport=0 0 400 300,bb=0 0 0 0,width=4cm,height=3cm,clip=true]{test.jpg}
        \caption{viewport=0 0 400 300\\bb=0 0 0 0}
        \label{essai_3}
    \end{subfigure}
    \caption{Les options pour l'insertion d'une image :\\viewport constant et bb inutile}%\label{fig-double}
    \label{viewport constant et bb inutile}

\end{figure}

\subsubsection{trim}
L'option trim fonctionne avec des coordonnées relatives.
Ainsi, {160 120 40 30} correspond à une zone partant du point {160 120}
et allant jusqu'au point situé à {-40 -30} par rapport au point haut droit.
Si l'image de départ a pour dimension {400 300}, alors effectivement, 
en coordonnées absolues, nous avons {160 120 360 270}.
Cela correspond a une image de 200x150.

Sur la figure~\ref{bb constant et trim variant}, page~\pageref{bb constant et trim variant},
nous avons, sur les deux premières lignes, un decoupage sans déformation de l'image, 
tandis que, sur la troisième ligne, nous avons un découpage avec déformation de l'image.

% debut d'une quatrième figure
\begin{figure}[h]
    \centering
    \begin{subfigure}[b]{0.3\textwidth}
        \includegraphics[bb=0 0 400 300,trim=0 0 0 0,width=4cm,height=3cm,clip=true]{test.jpg}
        \caption{bb=0 0 400 300\\trim=0 0 0 0}
        \label{essai_4}
    \end{subfigure}
    ~
    \begin{subfigure}[b]{0.3\textwidth}
        \includegraphics[bb=0 0 400 300,trim=160 120 40 30,width=4cm,height=3cm,clip=true]{test.jpg}
        \caption{bb=0 0 400 300\\trim=160 120 40 30}%
        \label{essai_5}
    \end{subfigure}
    ~
    \begin{subfigure}[b]{0.3\textwidth}
        \includegraphics[bb=0 0 400 300,trim=160 120 80 60,width=4cm,height=3cm,clip=true]{test.jpg}
        \caption{bb=0 0 400 300\\trim=160 120 80 60}%
        \label{essai_6}
    \end{subfigure}
    \\
    \begin{subfigure}[b]{0.3\textwidth}
        \includegraphics[bb=0 0 400 300,trim=160 120 120 90,width=4cm,height=3cm,clip=true]{test.jpg}
        \caption{bb=0 0 400 300\\trim=160 120 120 90}%
        %\includegraphics[viewport=0 0 400 300,bb=0 0 400 300,trim=0 0 0 0,width=4cm,height=3cm,clip=true]{test.jpg}
        %\caption{viewport=0 0 400 300\\bb=0 0 400 300\\trim=0 0 0 0}
        \label{essai_4}
    \end{subfigure}
    ~
    \begin{subfigure}[b]{0.3\textwidth}
        \includegraphics[bb=0 0 400 300,trim=160 120 160 120,width=4cm,height=3cm,clip=true]{test.jpg}
        \caption{bb=0 0 400 300\\trim=160 120 160 120}%
        %\includegraphics[viewport=0 0 400 300,bb=0 0 400 300,trim=50 50 60 60,width=4cm,height=3cm,clip=true]{test.jpg}
        %\caption{viewport=0 0 400 300\\bb=0 0 400 300\\trim=50 50 60 60}
        \label{essai_5}
    \end{subfigure}
    ~
    \begin{subfigure}[b]{0.3\textwidth}
        \includegraphics[bb=0 0 400 300,trim=160 120 200 150,width=4cm,height=3cm,clip=true]{test.jpg}
        \caption{bb=0 0 400 300\\trim=160 120 200 150}%
        %\includegraphics[viewport=0 0 400 300,bb=0 0 400 300,trim=100 100 110 110,width=4cm,height=3cm,clip=true]{test.jpg}
        %\caption{viewport=0 0 400 300\\bb=0 0 400 300\\trim=100 100 110 110}
        \label{essai_6}
    \end{subfigure}
    \\
    \begin{subfigure}[b]{0.3\textwidth}
        \includegraphics[bb=0 0 400 300,trim=160 120 160 90,width=4cm,height=3cm,clip=true]{test.jpg}
        \caption{bb=0 0 400 300\\trim=160 120 160 90}%deformation
        %\includegraphics[viewport=0 0 400 300,bb=0 0 400 300,trim=0 0 0 0,width=4cm,height=3cm,clip=true]{test.jpg}
        %\caption{viewport=0 0 400 300\\bb=0 0 400 300\\trim=0 0 0 0}
        \label{essai_7}
    \end{subfigure}
    ~
    \begin{subfigure}[b]{0.3\textwidth}
        \includegraphics[bb=0 0 400 300,trim=160 120 160 120,width=4cm,height=3cm,clip=true]{test.jpg}
        \caption{bb=0 0 400 300\\trim=160 120 160 120}
        %\includegraphics[viewport=0 0 400 300,bb=0 0 400 300,trim=50 50 60 60,width=4cm,height=3cm,clip=true]{test.jpg}
        %\caption{viewport=0 0 400 300\\bb=0 0 400 300\\trim=50 50 60 60}
        \label{essai_8}
    \end{subfigure}
    ~
    \begin{subfigure}[b]{0.3\textwidth}
        \includegraphics[bb=0 0 400 300,trim=160 120 160 150,width=4cm,height=3cm,clip=true]{test.jpg}
        \caption{bb=0 0 400 300\\trim=160 120 160 150}
        %\includegraphics[viewport=0 0 400 300,bb=0 0 400 300,trim=100 100 110 110,width=4cm,height=3cm,clip=true]{test.jpg}
        %\caption{viewport=0 0 400 300\\bb=0 0 400 300\\trim=100 100 110 110}
        \label{essai_9}
    \end{subfigure}
    \caption{Les options pour l'insertion d'une image :\\bb constant et trim variant}%\label{fig-double}
    \label{bb constant et trim variant}

\end{figure}


% debut d'une cinquieme figure
\begin{figure}[h]
    \centering
    \begin{subfigure}[b]{0.3\textwidth}
        \includegraphics[bb=0 0 400 300,trim=0 0 0 0,width=4cm,height=3cm,clip=true]{test.jpg}
        \caption{bb=0 0 400 300\\trim=0 0 0 0}
        \label{essai_a}
    \end{subfigure}
    ~
    \begin{subfigure}[b]{0.3\textwidth}
        \includegraphics[bb=160 120 560 420,trim=0 0 0 0,width=4cm,height=3cm,clip=true]{test.jpg}
        \caption{bb=160 120 560 420\\trim=0 0 0 0}
        \label{essai_2}
    \end{subfigure}
    ~
    \begin{subfigure}[b]{0.3\textwidth}
        \includegraphics[bb=160 120 400 300,trim=0 0 0 0,width=4cm,height=3cm,clip=true]{test.jpg}
        \caption{bb=160 120 400 300\\trim=0 0 0 0}
        \label{essai_3}
    \end{subfigure}
    \caption{Les options pour l'insertion d'une image :\\bb inutile et trim constant}%\label{fig-double}
    \label{bb inutile et trim constant}

\end{figure}


% debut d'une troisieme figure
\begin{figure}[!h]
\centering
%\includegraphics[width=\textwidth]{mydessin.pdf}
\includegraphics[width=300pt]{mydessin.pdf}
\caption{Ceci est encore mydessin.pdf}
\label{mydessin3}
\end{figure}

% debut d'une quatrieme figure
\begin{figure}[!h]
\centering
%\includegraphics[width=\textwidth]{mydessin.pdf}
\includegraphics[width=400pt]{mydessin.pdf}
\caption{Ceci est encore et toujours mydessin.pdf}
\label{mydessin4}
\end{figure}

% reference a une figure
Sur la figure~\ref{mydessin1}, page~\pageref{mydessin1}, nous avons mis une largeur de 100 pt.

Sur la figure~\ref{mydessin2}, page~\pageref{mydessin2}, nous avons mis une largeur de 200 pt.

Sur la figure~\ref{mydessin3}, page~\pageref{mydessin3}, nous avons mis une largeur de 300 pt.

Sur la figure~\ref{mydessin4}, page~\pageref{mydessin4}, nous avons mis une largeur de 400 pt.

% liste des figures
\listoffigures

\end{document}
