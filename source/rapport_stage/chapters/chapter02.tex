\section{Matériels utilisés}
Les mesures ont été réalisées à l'aide d'une station totale (théodolite robotisé) à visé sans prisme (ce qui permet de collecter des mesures de points inaccessibles).

Avec ce type d'appareil, l'opérateur observe un point avec sa lunette, point optique qui devient aussi la cible d'un laser, et dont les coordonnées géométriques peuvent être calculées.

\section{Les Conditions d'observations}
Afin de pouvoir observer plusieurs points sur la façade, l'appareil a été positionné en 5 lieux distincts.
Ces lieux sont appelés des stations d'observation.

Pour chacune de ces stations d'observation, l'opérateur a visé des points caractéristiques, c'est à dire qui peuvent être identifiés facilement d'une fois sur l'autre.
Cependant, aucun objet n'a été fixé sur la façade : les points caractéristiques utilisés sont donc des angles de pierre, des points particuliers d'une statue, mais aussi des vis servant à fixer des plaques aux fenêtres.
Le filet de protection accroché à l'échafaudage forme un écran visuel gênant pour l'observation de ces détails.

Deux stations d'observation ont été effectuées dans la cour d'honneur et trois autres stations ont été effectuées sur le chemin de ronde.

\section{La précision géométrique}

La précision des observations est de l'ordre de 2 à 3 millimètres.
Les erreurs de mesures sont dues en partie à l'instrument utilisé mais aussi et surtout à la qualité (netteté) des points observés.

Cette précision peut-être améliorée :
\begin{itemize}
  \item en utilisant le même type d'instrument mais en facilitant l'observation (utilisation de cibles fixés à la façade, suppression du filet de protection dans zone de la façade à surveiller, utilisation d'embase fixe pour les stations d'observations, ...)
  \item en utilisant un instrument de mesure permettant d'effectuer des mesures plus nombreuses ou plus précises (scanner ?)
\end{itemize}

Afin de déterminer et minimiser les erreurs de mesures ainsi que les erreurs de calculs, un point peut être observé plusieurs fois, à partir de deux ou trois stations différentes.

\section{Des Points fixes et des points à surveiller}

Le but est de mesurer des mouvements relatifs entre des points "fixes", disposés hors de la façade, et des points "cibles" situés sur la façade à surveiller.
Aussi, de chacune des stations d'observation, l'opérateur a aussi visé des points caractéristiques dans la cour ou sur le mur d'enceinte.

