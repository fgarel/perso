\section{Structure de l'entreprise}
Le port se divise en plusieurs « bloc » comme 
la capitainerie,
le service administratif et financier,
le service de comptabilité,
le service des infrastructures,
le service communication,
le service de la relation avec les clients
Tout cela a des directeurs qui sont eux même dirigé par un directeur général.
Mais il y a aussi 2 mission qui sont à par mais sous le commandement du directeur général
qui sont la mission du développement durable et la mission stratégie développement et aménagement.
Pour plus de précision se conférer à l'organigramme présent à la fin du dossier.


\section{Condition de travail}
Dans le port les services autre que les officier de quart ont des horaires normaux 
mais les officier de quart font des quart de 12h pour être là 24h/24h et 365j/365j 
car il n'est pas rare qu'un bateau arrive en pleine nuit.
Les officiers de quart font 3 quart par semaine donc il font 36 heures de travail par semaine
et les autres personnes font, elles, 35heures par semaine.
Les personnes de la capitainerie ont 25 jours de congés et en plus de jours d'anciennetés
au bout de plusieurs année passé dans la capitainerie,
dans la vigie les officier de quart ont les même « bonus » d’anciennetés mais ont 30 jours de congés directement.
Le travail de se déroule tout seul mais il y a souvent de la communication entre le personnel
car les officier de placement peuvent remplacer l'officier de quart quand celui-ci va aider
le pilote à accoster un navire.
Et il faut se parler quand il y a un problème sur un quai car c'est souvent l'officier de quart qui le sais
comme il est sur place donc l'officier de placement demande s'il n'y a pas de problème
avant de placer un cargo à l'officier de quart et celui-ci
lui dit si c'est possible ou pas mais la plupart du temps celui-ci dit s'il y a un problème
ou le marque sur un papier.
Le travail d'un officier de quart est souvent varié quand il y a beaucoup de navires
car ceux- là font des régulièrement des exercices d’alerte à la bombe, 
d’incendie ou autre que l'officier va gérer en prévenant les autorités, les pompiers,
et les autres personnes susceptible de devoir venir, celui-ci va aussi devoir les faire accoster,
ou faire partir ou devoir gérer des situations de crise
comme quand des personnes viennent dans le port sans leur autorisation et autre.
Les officiers de placement ont aussi un travail varier car ils doivent
placer des cargo qui ont jamais la même taille et sur des quais différents
car sur certain quai ont ne peut charger des céréales mais du bois ou inversement,
il doit prendre en compte plein de paramètre différent ce qui fait que chaque placement est différent.
Le travail de l'officier de placement est lui peu dangereux car il reste
dans son bureau mais celui de l'officier de quart l'est un peu plus
car quand celui-ci sort aider les pilotes à amarrer,
il est sur les quai où il peut tomber dans l'eau même si cela arrive que très rarement et il doit rester debout et attendre,
quelque soit le temps, que le bateau soit bien placé.



\section{Le recrutement}

Toute le personnel de la capitainerie a une expérience dans la marine
qu'elle soit marchande ou militaire car pour passer le concours il
faut avoir passé 3ans dans l'une ou l'autre avec au moins le diplôme d'officier de quart.
Si la personne qui passe le concours est première elle peut choisir
la capitainerie où elle veut aller en fonction des places disponibles
et c'est pareil pour le 2ème, le 3ème, etc et cela jusqu'à qu'il n'y ai plus de place disponible dans des capitaineries.
Pour ce métier il y a les même exigences que pour être officier dans la marine :
savoir bien parler anglais, être bon en math et en physique et savoir utiliser les instrument de navigation pour tous.
Et être en bonne forme physiquement pour les officier de quart car il faut aller dehors souvent.

\section{Les syndicats}

Dans la capitainerie, il y a 2 représentants du personnel.

