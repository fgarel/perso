
\section{Historique}

%\usepackage{wrapfig}
\begin{wrapfigure}{r}{0.5\textwidth}
    \begin{center}
        \includegraphics[width=0.48\textwidth]{P1020712.JPG}
        \end{center}
    \caption{A gull}
\end{wrapfigure}

%%%%%%%%%%%%%%%%%%%%%%%%%%%%%%%%%%%%%%%%%%%%%%%%%%%%%%%%%
% 3 parametres :
Tout d'abord le port fut construit à la fin du 19° siècle à la demande de grand armateur de La Rochelle 
car le vieux port qui était port de commerce était trop petit.
Donc celui-ci fut creusé à la Pallice car à l'époque de la construction ce n'était qu'un petit village.
Le port atlantique de La Rochelle fut inauguré le 19 août 1890 mais les travaux, 
confiés à l'ingénieur Bouquet de la Grye en 1876 commencèrent en 1880.
Puis le port accueilli multiple travaux de rénovation ou d'agrandissement plus ou moins grand.
Tout ces travaux ont agrandit la superficie du port qui a maintenant 310ha d'espace terrestre et 233ha
d'espace maritime et donc 543ha d'espace total.
De l'inauguration au 1er janvier 2006 le port appartenait à la chambre de commerce et de l’industrie de La Rochelle.
Puis l'état le racheta et il devenu port autonome.
Puis suite à un décret il devenu grand port maritime le 9 octobre 2008.
Les terrains appartiennent à l'état donc au port et celui-ci loue des terrains
aux entreprises qui souhaitent profiter de la proximité du port pour un bail s’élevant à plusieurs années.

\section{Nature des activités}
Le port atlantique de La Rochelle produit des services tout en étant une administration
donc elle fait partie du secteur tertiaire.
Car dans celle ci il y a la capitainerie, que j'ai observé, qui représente l'autorité portuaire
et le placement des cargo ou bateau plus petit ce qui est un travail d'envergure
car dans le port comporte 4264mêtres de quais,
il y a  plusieurs endroits où le niveau d'eau n'est pas le même et la marée est différente chaque jour.
Tout cet activité sont des services car c'est beaucoup d'aide au armateur et au capitaine de navire.
La capitainerie vérifie aussi pour eux les documents obligatoires pour enter dans un port
et aide les pilotes et les capitaines de cargo à accoster.

\section{Statut juridique}
Le Port Atlantique de La Rochelle est une entreprise publique car celle ci appartient à l'état.

\section{Taille de l'entreprise}
Le Port Atlantique de La Rochelle emploi 100 personnes mais si on compte les emplois 
qui se trouvent dans le port et dons pas que ceux employé par l'état,
le port emploi environ 1 600 emplois.
Et si on compte les emplois directs, indirects et induits le chiffre s'élève à environ 16 300 emplois.


